% Template LaTeX file for DAFx-14 papers
%
% To generate the correct references using BibTeX, run
%     latex, bibtex, latex, latex
% modified...
% - from DAFx-00 to DAFx-02 by Florian Keiler, 2002-07-08
% - from DAFx-02 to DAFx-03 by Gianpaolo Evangelista
% - from DAFx-05 to DAFx-06 by Vincent Verfaille, 2006-02-05
% - from DAFx-06 to DAFx-07 by Vincent Verfaille, 2007-01-05
%                          and Sylvain Marchand, 2007-01-31
% - from DAFx-07 to DAFx-08 by Henri Penttinen, 2007-12-12
%                          and Jyri Pakarinen 2008-01-28
% - from DAFx-08 to DAFx-09 by Giorgio Prandi, Fabio Antonacci 2008-10-03
% - from DAFx-09 to DAFx-10 by Hannes Pomberger 2010-02-01
% - from DAFx-10 to DAFx-12 by Jez Wells 2011
% - from DAFx-12 to DAFx-14 by Sascha Disch 2013
%
% Template with hyper-references (links) active after conversion to pdf
% (with the distiller) or if compiled with pdflatex.
%
% 20060205: added package 'hypcap' to correct hyperlinks to figures and tables
%                      use of \papertitle and \paperauthorA, etc for same title in PDF and Metadata
%
% 1) Please compile using latex or pdflatex.
% 2) If using pdflatex, you need your figures in a file format other than eps! e.g. png or jpg is working
% 3) Please use "paperftitle" and "pdfauthor" definitions below

%------------------------------------------------------------------------------------------
%  !  !  !  !  !  !  !  !  !  !  !  ! user defined variables  !  !  !  !  !  !  !  !  !  !  !  !  !  !
% Please use these commands to define title and author of the paper:
\def\papertitle{Templates for DAFx-15, Trondheim, Norway}
\def\paperauthorA{Vincent Lostanlen}
\def\paperauthorB{St\'{e}phane Mallat}
\def\paperauthorC{Joseph Fourier}
\def\paperauthorD{Claude Shannon}

\hyphenation{Facto-rization}

%------------------------------------------------------------------------------------------
\documentclass[twoside,a4paper]{article}
\usepackage{dafx_15}
\usepackage{amsmath,amssymb,amsfonts,amsthm}
\usepackage{euscript}
\usepackage[latin1]{inputenc}
\usepackage[T1]{fontenc}
\usepackage{ifpdf}

\usepackage[english]{babel}
\usepackage{caption}
\usepackage{subfig, color}

\setcounter{page}{1}
\ninept

\usepackage{times}
% Saves a lot of ouptut space in PDF... after conversion with the distiller
% Delete if you cannot get PS fonts working on your system.

% pdf-tex settings: detect automatically if run by latex or pdflatex
\newif\ifpdf
\ifx\pdfoutput\relax
\else
   \ifcase\pdfoutput
      \pdffalse
   \else
      \pdftrue
\fi

\ifpdf % compiling with pdflatex
  \usepackage[pdftex,
    pdftitle={\papertitle},
    pdfauthor={\paperauthorA, \paperauthorB, \paperauthorC, \paperauthorD},
    colorlinks=false, % links are activated as colror boxes instead of color text
    bookmarksnumbered, % use section numbers with bookmarks
    pdfstartview=XYZ % start with zoom=100% instead of full screen; especially useful if working with a big screen :-)
  ]{hyperref}
  \pdfcompresslevel=9
  \usepackage[pdftex]{graphicx}
  \usepackage[figure,table]{hypcap}
\else % compiling with latex
  \usepackage[dvips]{epsfig,graphicx}
  \usepackage[dvips,
    colorlinks=false, % no color links
    bookmarksnumbered, % use section numbers with bookmarks
    pdfstartview=XYZ % start with zoom=100% instead of full screen
  ]{hyperref}
  % hyperrefs are active in the pdf file after conversion
  \usepackage[figure,table]{hypcap}
\fi

\title{\papertitle}

%-------------SINGLE-AUTHOR HEADER STARTS (uncomment below if your paper has a single author)-----------------------
\affiliation{
\paperauthorA, \paperauthorB \sthanks{This work was supported by the XYZ Foundation}}
{\href{http://di.ens.fr/data/}{Dept. of Computer Science}, \\
\'{E}cole normale sup\'{e}rieure \\
Paris, France\\
{\tt \href{mailto:vincent.lostanlen@ens.fr}{vincent.lostanlen@ens.fr}}
}
%-----------------------------------SINGLE-AUTHOR HEADER ENDS------------------------------------------------------

%---------------TWO-AUTHOR HEADER STARTS (uncomment below if your paper has two authors)-----------------------
%\twoaffiliations{
%\paperauthorA, \sthanks{This work was supported by the XYZ Foundation}}
%{\href{http://dafx14.fau.de}{Fraunhofer IIS,} \\ Friedrich-Alexander University Erlangen\\ Erlangen, Germany\\
%{\tt \href{mailto:dafx14@audiolabs-erlangen.de}{dafx14@audiolabs-erlangen.de}}
%}
%{\paperauthorB, \sthanks{This guy is a very good fellow}}
%{\href{http://music.nuim.ie/}{Dept. of Music,} \\ National University of Ireland\\ Maynooth, Ireland\\
%{\tt \href{mailto:papers@dafx13.nuim.ie}{papers@dafx13.nuim.ie}}
%}
%-------------------------------------TWO-AUTHOR HEADER ENDS------------------------------------------------------

%---------------THREE-AUTHOR HEADER STARTS (uncomment below if your paper has three authors)-----------------------
%\threeaffiliations{
%\paperauthorA, \sthanks{This work was supported by the XYZ Foundation}}
%{\href{http://dafx14.fau.de}{Fraunhofer IIS,} \\ Friedrich-Alexander University Erlangen\\ Erlangen, Germany\\
%{\tt \href{mailto:dafx14@audiolabs-erlangen.de}{dafx14@audiolabs-erlangen.de}}
%}
%{\paperauthorB, \sthanks{This guy is a very good fellow}}
%{\href{http://music.nuim.ie/}{Dept. of Music,} \\ National University of Ireland\\ Maynooth, Ireland\\
%{\tt \href{mailto:papers@dafx13.nuim.ie}{papers@dafx13.nuim.ie}}
%}
%{\paperauthorC,\sthanks{Illustrious contributor}}
%{\href{http://dafx14.fau.de}{Spectral Labs,} \\ Somewhere above the sky\\ {\tt \href{mailto:spectrum@sinus.oids}{mailto:spectrum@sinus.oids}}
%}
%-------------------------------------THREE-AUTHOR HEADER ENDS------------------------------------------------------

%----------------FOUR-AUTHOR HEADER STARTS (uncomment below if your paper has four authors)-----------------------
%\fouraffiliations{
%\paperauthorA, \sthanks{This work was supported by the XYZ Foundation}}
%{\href{http://dafx14.fau.de}{Fraunhofer IIS,} \\ Friedrich-Alexander University Erlangen\\ Erlangen, Germany\\
%{\tt \href{mailto:dafx14@audiolabs-erlangen.de}{dafx14@audiolabs-erlangen.de}}
%}
%{\paperauthorB, \sthanks{This guy is a very good fellow}}
%{\href{http://music.nuim.ie/}{Dept. of Music,} \\ National University of Ireland\\ Maynooth, Ireland\\
%{\tt \href{mailto:papers@dafx13.nuim.ie}{papers@dafx13.nuim.ie}}
%}
%{\paperauthorC,\sthanks{Illustrious contributor}}
%{\href{http://dafx14.fau.de}{Spectral Labs,} \\ Somewhere above the sky\\ {\tt \href{mailto:spectrum@sinus.oids}{mailto:spectrum@sinus.oids}}
%}
%{\paperauthorD,\sthanks{Yes, senior}}
%{\href{http://www.audiolabs-erlangen.com}{Communications Office, Somewhere else,} \\ Centre for Something \\ {\tt \href{mailto:some@where.no }{some@where.no}}
%}
%-------------------------------------FOUR-AUTHOR HEADER ENDS------------------------------------------------------

\begin{document}
% more pdf-tex settings:
\ifpdf % used graphic file format for pdflatex
  \DeclareGraphicsExtensions{.png,.jpg,.pdf}
\else  % used graphic file format for latex
  \DeclareGraphicsExtensions{.eps}
\fi

\maketitle

\begin{abstract}
\end{abstract}

\section{Introduction}
Spectrogram-based pattern recognition algorithms, such as sparse coding [Abdallah Plumbley 2005] and Nonnegative Matrix Factorization [Smaragdis Brown 2003], are widespread in audio signal processing.
They are designed to approximate their input by a linear combination of few data-driven templates.
Musical chords, for example, are expected to get decomposed into individual notes.

However, most natural sounds cannot be factorized as amplitude-modulated fixed spectra: notably, continuous changes in pitch (e.g. vibrato, glissando) as well as in spectral envelope (e.g. attack transients, formantic transitions) have a joint time-frequency structure that cannot be matched to a single spectral atom.
Time-varying, under-constrained generalizations have been devised to address this shortcoming [Hennequin et al. 2011], but their high number of parameters prevents their robustness in challenging polyphonic contexts.

Instead of specifying probabilistic priors to help the convergence [Fuentes et al. 2013], we aim to design a template-free, nonlinear, mid-level representation, that natively disentangles the time variabilities of pitch and spectral envelope.

The central idea to our representation is that the former correspond to rigid motions along the log-frequency axis, whereas the latter affect the relative amplitude of harmonics across neighboring octaves.
This distinction can be conceptually emphasized by arranging the log-frequency axis in a spiral, hence aligning frequency bins that share the same "chroma", i.e. musical pitch class [Shepard 1964].
By means of a multivariable wavelet transform (see Fig. 1), which consists of joint time-chroma-octave convolutions, changes in pitch and spectral envelope are respectively captured as angular and radial motions on the spiral.

The contributions of this paper are:
\begin{itemize}
\item
the introduction of the Shepard spiral scattering transform as a cascade of wavelet operators,
\item
a nonstationary formulation of the source-filter convolutional model relying on time warps, and its factorization in the wavelet scalogram,
\item
an approximate closed-form expression of Shepard spiral scattering coefficients, showing that variabilities in pitch and spectral envelope get jointly linearized, and stably appear as energy maxima.
\item
a visualization of these coefficients in Berio's \emph{Sequenza V}, revealing extended instrumental techniques.
\end{itemize}

\section{Shepard spiral scattering}

Let $\psi(t)=\vert\psi\vert(t)\mathrm{e}^{2\pi\mathrm{i}t}$ a "mother wavelet" of dimensionless center frequency $1$ and bandwidth $Q^{-1}$. The quality factor $Q$ is an integer in the typical range $12$\textendash$24$.
Center frequencies of the subsequent wavelet filter bank are of the
form $\lambda_{1} = 2^{j_{1} + \frac{\chi_{1}}{Q}}$, where the indices
$j_{1} \in \mathbb{Z}$ and $\chi_1 \in \{1\ldots\,Q\}$ respectively denote
octave and chroma.
\[
\psi_{\lambda_{1}}(t)=\lambda_{1}\psi(\lambda_{1}t)\quad\mathrm{i.e.}\quad\widehat{\psi_{\lambda_{1}}}(\omega)=\widehat{\psi}(\lambda^{-1}\omega)
\]

The wavelet transform of an audio signal $x(t)$ is defined as the array of convolutions $x \ast \psi_{\lambda_1}(t)$ for every audible frequency $\lambda_1$. The modulus of the resulting signals, called \emph{scalogram}, localize the power spectrum of $x(t)$ around the log-frequencies $\log_2 \lambda_1 = j_1 + \frac{\chi_1}{Q}$ over durations $2 Q \lambda_1^{-1}$, trading frequency resolution for time resolution:

\[
x_1 (t, \log_{2}\lambda_{1}) = \left| x \ast \psi_{\lambda_{1}} \right| (t).
\]

The scattering transform has been introduced in signal classification to achieve discriminative invariants to time shifts, while ensuring stability to small time warps. Invariance up to some maximal time shift $T$ is achieved with a lowpass filter $\phi_T$ of time support $T$. The outcome of this filtering are the constant-Q transform (CQT) coefficients $S_1 x$, indexed by time $t$ and log-frequency $\log_2 \lambda_1$:

\[
S_1 x (t, \log_2 \lambda_1) = x_1 \ast \phi_T (t).
\]

This averaging discards all variability in $x_1$ at finer time scales than $T$, that is, higher modulation frequencies than $T^{-1}$. To recover them, the scattering transform convolves $x_1$ with a second filterbank of wavelets $\psi_{\lambda_2}(t)$ of center frequencies $\lambda_2$, and applies complex modulus to improve regularity:

\[
x_2 (t, \log_2 \lambda_1, \log_2 \lambda_2) = \left| x_1 \ast \psi_{\lambda_{2}} \right| (t)
\]

Also known as \emph{amplitude modulation spectrum} three-way array $x_2$

$S_2 x (t, \log_2 \lambda_1, \log_2 \lambda_2) = x_2 \ast \phi_T (t)$



%\newpage
\nocite{*}
\bibliographystyle{IEEEbib}
\bibliography{DAFx15_tmpl} % requires file DAFx15_tmpl.bib
 
\end{document}
